\documentclass[aspectratio=169]{beamer}
%\documentclass{beamer}
%%%CHOOSE ASPECT RATIO ABOVE%%%

\usetheme{LU}

\usepackage[utf8]{inputenc}
\usepackage{csquotes}
\usepackage[british]{babel}
\usepackage{graphicx}
% \usepackage{booktabs}
\usepackage{makecell}
\usepackage{textcomp}
\usepackage{listings}

\renewcommand\theadfont{\tiny}
\renewcommand*{\bibfont}{\scriptsize}

% ------------------------------------------------------------------------------

\title[Causal Inference]{Difference-in-Differences \& \newline two-way fixed effects}

\titlecolor{LUIvory} % Choose between LUPink, LULBlue, LUIvory, LUGreen
\titleimage{\includegraphics[scale=.955]{Grayscale-Globe.jpg}}
\author{Nils Droste}
\subtitle{}
\date{dd mm yyyy}
\institute{Lund University\\Department for Political Science}
\newcommand{\conference}{2023 ClimBEco course}

% bibliography
\addbibresource{C:/Users/NILSDR~1/Dropbox/Dokumente/references/library.bib}

\begin{document}

% ------------------------------------------------------------------------------

\titleframe


% ------------------------------------------------------------------------------
% \section{Outline}
	%
	% \begin{frame}{Advanced Quantitative Analysis Part}
	% 	\begin{itemize}
	% 		\item \textit{\textbf{1 voluntary lecture:}}
	% 		\begin{itemize}
	% 			\item Causal inference with observational data (today)
	% 		\end{itemize}
	% 		\item \textit{\textbf{2 seminars:}}
	% 		\begin{itemize}
	% 			\item Panel Data and IV (Thu 10: 10-12h \& 13-15h),
	% 			\item Regression Discontinuity Design (Thu 17: 10-12h \& Thu 17: 13-15h)
	% 		\end{itemize}
	% 		\item \textit{\textbf{2.5 assignments:}}
	% 		\begin{itemize}
	% 			\item Draft research proposal, $\sim$2 pages (\textcolor{red}{\textbf{due Dec 18}}),
	% 			\begin{itemize}
	% 				\item 2 $\times$ $\sim$150 word peer review (\textcolor{red}{\textbf{due Jan 8}})
	% 			\end{itemize}
	% 			\item Final research proposal (\textcolor{red}{\textbf{due Jan 17}})
	% 		\end{itemize}
	% 	\end{itemize}
	% \end{frame}

% ------------------------------------------------------------------------------
\section{Introduction}


	\begin{frame}{Standing on the shoulders of giants}
		\begin{centering}
			\includegraphics[scale=.33]{nobel-economics.jpg}
			\captionof{figure}{\textcolor{gray}{Card, Angrist \& Imbens won the 2021 Nobel Prize in Economics for causal analysis. \\
			See: \href{https://www.nobelprize.org/prizes/economic-sciences/2021/summary/}{\underline{\smash{Sverige Riksbank}}}}}
		\end{centering}
	\end{frame}

	\begin{frame}{Design vs. model-based causal inference}
		\begin{itemize}
			\item \textit{\textbf{design-based}}: properties of the estimators arises from random treatment assignment
			\item \textit{\textbf{model-based}}: properties arise from the random sampling of units from a large population (in combination with assumptions on this population distribution.)
		\end{itemize}
		\vspace*{.5cm} From \cite{Athey2022}
	\end{frame}

	\begin{frame}{Design vs. model-based causal inference - 2}
		\begin{itemize}
			\item there is different takes on which approach allows causal estimates cf. (\cite{Card2022, Heckman2008})
			\item finite populations without random sampling are tricky, especially when there is no random treatment (\cite{Sterba2009, Abadie2020})
			\item here, structural causal models can help to explicitely formulate the data generating process (but not just in such cases) (\cite{Cinelli2022})
		\end{itemize}
	\end{frame}

	\begin{frame}{Three tasks in the quantitive analysis of causes}
		\begin{centering}
			\begin{tabular}{cp{6.5cm}p{3cm}}
				Task & Description & Requirements \\
				\hline \hline \\
					1 & Defining the set of hypotheticals or counterfactuals & A scientific \textbf{theory} \\\\
					2 & Identifying causal parameters from hypothetical population data & Mathematical analysis of point or set  \textbf{identification} \\\\
					3 & Identifying parameters from real data &  \textbf{estimation} and testing theory \\
			\end{tabular}
		\end{centering}
		\\\vspace*{.5cm} From \cite{Heckman2008}
	\end{frame}

	\begin{frame}{Causal Inference from observational data}
		\textbf{\textit{Synopsis}}: Today, we will be looking into \textit{the} classical research design for inferring causal effects from observational data (i.e. when experiments are unethical or infeasible), and its recent developments \\ \vspace*{.25cm}
		\onslide<2->{In particular, we will develop an understanding of\\ \vspace*{.25cm}}
		\begin{itemize}
			\item<3-> quasi / natural experiments
			\begin{itemize}
				\item<4-> Difference-in-Differences
				\item<5-> (two-way) fixed-effects regressions
				\item<6-> staggered treatment
			\end{itemize}
		\end{itemize}
	\end{frame}

	\begin{frame}{Definition}
		\textbf{Quasi- / natural experiment:} \\ \vspace*{.25cm}
		A setting where a subpopulation is treated with an intervention of sorts that occurs due to non-random assignment processes (outside of the researchers influence if its called \textit{natural}). \\ \\
		\onslide<2>{\textbf{However:} "First, it is worth noting that the label “natural experiment” is perhaps unfortunate. As we shall see, the social and political forces that give rise to as-if random assignment of interventions are not generally “natural” in any ordinary sense of that term. 15 Second, natural experiments are observational studies, not true experiments, again, because they lack an experimental manipulation. In sum, natural experiments are neither natural nor experiments. (\cite{Dunning2012})}
	\end{frame}

% ------------------------------------------------------------------------------
\section{Diff-in-Diff}

	\begin{frame}{Recall potential outcome approximation}
		We may choose to infer an average treatment effect (ATE) $I$: i.e. $ I =\{A,B...\} $ by comparing the average outcomes of treated individuals $a$ from $A$ with the one of untreated individuals $b \in B$: \\
		\begin{equation}
			E \{\Delta \textcolor{red}{Y}_i \}  \approx  E \{\textcolor{red}{Y}_a(\textcolor{blue}{1})\} - E \{\textcolor{red}{Y}_b(\textcolor{blue}{0})\}
		\end{equation} \\
		For such a case we can exploit \textit{random chance} within sufficiently large samples to make these groups comparable. \\ \vspace*{.25cm}
		\onslide<2> \textbf{\textit{But what if we do not have a random assignment}} (and there may be a selection-bias and / or substantial differences between groups)?
	\end{frame}

	\subsection{intuition}
		\begin{frame}{a famous case}
			By comparing a treated with an untreated group over 2+ periods, we can control for (time-constant) differences between groups.
			\\ \vspace*{.05cm}
			\onslide<2->{A classic example is the Card and Krueger (AER, 1994), comparing fast-food worker employment in Pennsylvania (PA) and New Jersey (NJ) before and after a minimum wage raise in NJ in 1992.}
			\begin{center}
				\onslide<2->\includegraphics[width=.5\textwidth]{CardKrueger2000.jpg}
				\\ \onslide<2-> \tiny{ Image source: \cite{Card2000}}
			\end{center}
		\end{frame}

		\begin{frame}{results}
			Here is a (tweaked) version of Card and Krueger \citeyear{Card1994}, Figure 1.
			\\ \vspace*{.05cm}
			\begin{center}
				\includegraphics[width=\textwidth]{CK94.png}
				\\\tiny{ Image source: \cite{Card1994}}
			\end{center}
		\end{frame}


	\subsection{potential outcome}
		\begin{frame}{potential outcome notation}
			Here, we are interested in the average treatment effect on the treated (ATT)

			\begin{equation}
				ATT = E[\textcolor{red}{Y}_i(\textcolor{blue}{1}) - \textcolor{red}{Y}_i(\textcolor{blue}{0}) | D = \textcolor{blue}1]
			\end{equation}
			\\ \vspace*{.05cm}
			\onslide<2-> For this to be a consistent estimator, we will need a set of conditions to hold, some of which we can test, others we will need to assume. \\ \vspace*{.1cm}
			\onslide<3-> In particular, the parallel-trends assumption:
			\begin{equation}
				\onslide<3->
				E[\textcolor{red}{Y}_{it}(\textcolor{blue}{0}) - \textcolor{red}{Y}_{it-1}(\textcolor{blue}{0})|D=\textcolor{blue}1] =E[\textcolor{red}{Y}_{it}(\textcolor{blue}{0}) - \textcolor{red}{Y}_{it-1}(\textcolor{blue}{0})|D=\textcolor{blue}0]
			\end{equation}
			\\ \vspace*{-.5cm}
			\onslide<4-> because then, we can assume
			\begin{equation}
				\onslide<4-> ATT=E[\textcolor{red}{Y}_t - \textcolor{red}{Y}_{t-1}|D=\textcolor{blue}{1}] - E[\textcolor{red}{Y}_t-\textcolor{red}{Y}_{t-1}|D=\textcolor{blue}{0}]
			\end{equation}
		\end{frame}

	\subsection{DAGs}
		\begin{frame}{Directed acyclic graphs}
			Difference-in-Differences (DID)
			\\ \vspace*{-.35cm}
			\begin{columns}
				\begin{column}{0.5\textwidth}
					\begin{center}
						\onslide<1->{\includegraphics[width=\textwidth]{dag-did-1}}
					\end{center}
				\end{column}
				\begin{column}{0.5\textwidth}
					\begin{center}
						\onslide<2->{\includegraphics[width=\textwidth]{dag-did-2}}
					\end{center}
				\end{column}
			\end{columns}
			\scriptsize Image source: \href{http://nickchk.com/causalgraphs.html}{\underline{\smash{Huntington-Klein 2018}}}
			\\ \vspace*{.25cm}
			\small \onslide<3> {$\rightarrow$ Accounting for differences between groups over time enables not just the estimation of time and group effects but also the differences-in-differences \\ (i.e. the interaction of time and group effects).}
		\end{frame}

	\subsection{estimation}
		\begin{frame}{Estimation}
			\begin{columns}
				\begin{column}{0.66\textwidth}
					\begin{center}
						\only<1>{\animategraphics[width=.7\textwidth,controls]{10}{DID-}{0}{149}}
						\only<2>{\includegraphics[width=.7\textwidth]{DID}}
					\end{center}
				\end{column}
				\begin{column}{0.33\textwidth}
					DID Estimator \\ \vspace*{.25cm}
					\footnotesize assume $n$ individual units $i$, and $t=2$ time periods, \\
					we can estimate the effect of a treatment ocurring at $P_{t=1}$,
					affecting the treated subpopulation $D_i=1$
					\begin{equation}
						\begin{split}
							Y_{i} = \beta_0 & + \beta_1 D_i + \beta_2 P_t \\
											& + \beta_3 D_i \times P_t + \varepsilon_{i}
						\end{split}
					\end{equation}
					\scriptsize{with $D_i = \text{Treatment}$, $P_t = \text{Period Dummy}$. \\ \vspace*{.25cm}
					\footnotesize $\beta_3$ gives us an estimate of the diff-in-diff treatment effect.}
				\end{column}
			\end{columns}
		\end{frame}

		% \begin{frame}{assumptions}
		% 	\begin{itemize}
		% 		\item parallel trends
		% 		\item SUTVA
		% 		\item random treatment assignment
		% 		\item non-zero average causal effect of Z on D
		% 		\item exclusion restriction (no backdoor from assignment to treatment)
		% 	\end{itemize}
		% \end{frame}

		\begin{frame}[fragile]{regression}
			To estimate an example ATT, we can use the Card and Krueger (1994) \href{http://economics.mit.edu/faculty/angrist/data1/mhe/card}{\underline{\smash{data}}}
			% \\ \vspace*{.25cm}
 			\begin{verbatim}
				> did_model <- lm(emptot ~ time + treated + time:treated,
				                  data = card_krueger_1994_mod)
				> summary(did_model)

				Coefficients:
				               Estimate Std. Error t value Pr(>|t|)
				(Intercept)    23.331      1.072  21.767   <2e-16 ***
				time           -2.166      1.516  -1.429   0.1535
				treated        -2.892      1.194  -2.423   0.0156 *
				time:treated    2.754      1.688   1.631   0.1033
 	 		\end{verbatim}
			\vspace*{-.5cm}
			\footnotesize \textit{Note}: heteroskedasticity and autocorrelation robust standard errors should be computed
		\end{frame}

		\begin{frame}{intermediate summary}
			A difference-in-differences approach allows us to \\ \vspace*{.5cm}
			\begin{itemize}
				\item compare a treatment group with an untreated quasi-counterfactual
				\item even under conditions of a non-random assignment
				\item assuming that the groups behave comparably enough
			\end{itemize}
		\end{frame}

% ------------------------------------------------------------------------------
\section{Fixed effects}

	\subsection{intuition}

		\begin{frame}{Panel Data}
			Let us suppose a situation with repeated measurements for multiple individual(s) (units), i.e. cross-sectional time-series, or panel data.

			\begin{table}
				\caption{An example panel data structure}
				\footnotesize
				\begin{tabular}{|p{1.8cm}|p{1.6cm}|p{1.6cm}|p{1.6cm}||p{1.6cm}|}
					\hline
					\textbf{individual (i)} & \textbf{time (t)} & $\textbf{Y}_{it}$ & $\textbf{X}_{it}$ & $\textbf{D}_{it}$ \\
					\hline
					A & 1 & 0.8 & 0.3 & 0 \\
					A & 2 & 0.7 & 0.2 & 0 \\
					A & 3 & 0.5 & 0.2 & 1 \\
					\hline
					B & 1 & 1.2 & 0.4 & 0 \\
					B & 2 & 1.1 & 0.5 & 0 \\
					B & 3 & 0.9 & 0.6 & 1 \\
					\hline
					{...} & {...} & {...} & {...} & {...} \\
					\hline
				\end{tabular}
			\end{table}
		\end{frame}

		\begin{frame}{Fixed-Effects}
			An exemplary study
			\begin{center}
				\only<1>{\includegraphics[scale=.2]{Binder}}
				\only<2>{\includegraphics[scale=.7]{Environmental_Kuznets_Curve}
								 \captionof{figure}{Theoretical underpinning for \cite{Binder2005}: Environmental Kuznets Curve. Image source: \href{https://en.wikipedia.org/wiki/Kuznets_curve}{\underline{\smash{Wikipedia}}}}}
			\end{center}
		\end{frame}

		\begin{frame}[fragile]{Fixed-Effects}
			Replicated results from table 2, model 1 Binder and Neumayer \citeyear{Binder2005}
			% \\ \vspace*{.25cm}
			\begin{verbatim}
				Coefficients:
				                   Estimate Std. Error t value Pr(>|t|)
				(Intercept)       -57.55012   16.60671  -3.465 0.000548 ***
				lnengopc           -0.51121    0.11878  -4.304 1.82e-05 ***
				lnenergy            1.00887    0.60455   1.669 0.095425 .
				lngdp              13.81819    4.17975   3.306 0.000975 ***
				lngdpsq            -0.88657    0.27384  -3.238 0.001239 **
				polity             -0.05079    0.03023  -1.680 0.093135 .
			\end{verbatim}
			% \vspace*{-.5cm}
			\footnotesize \textit{Note}: This is not a fixed effects regression but an OLS
		\end{frame}

	\subsection{DAGs}

		\begin{frame}{Panel Data}
			One way fixed-Effects (FE) structure
			\begin{center}
				\includegraphics[scale=.7]{dag-fe}
				\captionof{figure}{Accounting for individual fixed effects. Image source: \href{http://nickchk.com/causalgraphs.html}{\underline{\smash{Huntington-Klein 2018}}}}
			\end{center}
		\end{frame}

		\begin{frame}{Fixed-Effects}
			\begin{columns}
				\begin{column}{0.66\textwidth}
					\begin{center}
				 		\only<1>{\animategraphics[width=.7\textwidth,controls]{10}{FixedEffects}{0001}{0200}}
						\onslide<2->{\includegraphics[width=.7\textwidth]{FEstatic.png}}
					\end{center}
				\end{column}
				\begin{column}{0.33\textwidth}
					\onslide<3>{
						FE Estimator
						\begin{equation}
							Y_{i} = \alpha_i + \beta X_{it} + \varepsilon_{it}
						\end{equation}
						\scriptsize{with $\alpha_i  = $ individual, time-invariant effect, and $X_i = $ a variable of interest.
						The so called \textit{within} transformation accounts for $\alpha_i$ through demeaning such that we can consistently estimate $\partial Y / \partial X$ }
					}
				\end{column}
			\end{columns}
		\end{frame}

		\begin{frame}{Fixed-Effects}
			Two way fixed-effects
			\begin{center}
				\includegraphics[scale=.7]{dag-fe-tw}
				\captionof{figure}{Accounting for individual and time (two-way) fixed effects. Image source: \href{http://nickchk.com/causalgraphs.html}{\underline{\smash{Huntington-Klein 2018}}}}
			\end{center}
		\end{frame}

	\subsection{estimation}

		\begin{frame}{Estimator}
			The two-way fixed effects model can be formulated as
			\\ \vspace*{.25cm}

			\begin{equation}
				Y_{it} = \alpha_i + \theta_t + \tau D_{it} + \beta X_{it} + \varepsilon_i
			\end{equation}

			\vspace*{.25cm}
			where $\alpha_i$ individual and $\theta_t$ time fixed effects, respectively \\
			\textbf{Note} that $D$ here is a binary treatment indicator ($D_{it}=D_{i} \times P_{t}$), \\ possibly a vector of control variables $X_i$, \\
			and error term $\varepsilon_i$, assumed to be normally distributed and centered around $0$, indepdendent of everything else.
		\end{frame}

		\begin{frame}{time-trends}
			This allows to take both individual differences and shocks in time into account
			\begin{center}
				\includegraphics[width=.7\textwidth]{CardKrueger2000b.jpg}
				\\ \tiny{ Image source: \cite{Card2000}}
			\end{center}
		\end{frame}

		\begin{frame}{intermediate summary}
			A two-way fixed effects approach allows us to \\ \vspace*{.5cm}
			\begin{itemize}
				\item control for time-constant, unobserved heterogeneity between individuals
				\item control for common time shocks, that affect all individuals
				\item a multi-period 2WFE approach resembles DiD when
				\begin{itemize}
					\item treatment is simultaneous
					\item effects are homogeneous
				\end{itemize}
			\end{itemize}
		\end{frame}

% ------------------------------------------------------------------------------
\section{Staggered treatment}

	\subsection{intuition}
		\begin{frame}{Staggered treatment and \\ other current developments}
			Over the last 6 or so years, DiD methdology has seen quite some development.
			\\ \vspace*{.25cm}
			I will briefly introduce
			\\ \vspace*{.25cm}
			\begin{itemize}
				\item<2-> staggered treatment / event-time studies
				\item<3-> non-parallel trend corrections
				\item<4-> heterogeneous treatment corrections
			\end{itemize}
		\end{frame}

		\begin{frame}{intuition}
			What if treatment happened at different times?
			\begin{center}
				\includegraphics[width=.6\textwidth]{BGB2015.jpg}
				\\ \tiny{Reduction in mortality by Community Health Centers (CHC). Image source: \cite{Bailey2015}}
			\end{center}
		\end{frame}

    % time fixed effects ???

		\begin{frame}{intuition}
			We may run into time series length related weighting problems
			\begin{center}
				\includegraphics[width=.7\textwidth]{StaggeredTreatment.png}
				\\ \tiny{DiD with variations in treatment timing. Image source: \cite{Goodman-Bacon2021}, as reproduced by \href{https://andrewcbaker.netlify.app/2019/09/25/difference-in-differences-methodology/}{\underline{\smash{A.C. Baker}}}}
			\end{center}
		\end{frame}

		\begin{frame}{intuition}
			Recall,
			\\ \vspace*{.25cm}
			\begin{center}
				\begin{math}
					ATT=E[\textcolor{red}{Y}_t - \textcolor{red}{Y}_{t-1}|D=\textcolor{blue}{1}] - E[\textcolor{red}{Y}_t-\textcolor{red}{Y}_{t-1}|D=\textcolor{blue}{0}]
				\end{math}
			\end{center}
			which is a comparison of mean differences between groups over time. Straightforward for 2 periods.
			\\ \vspace*{.25cm}
			Now, we have multiple periods.
		\end{frame}

		\begin{frame}{intuition}
			Picture a case with multiple periods, three groups, and thus four comparisons
			\begin{center}
				\includegraphics[width=.7\textwidth]{StaggeredTreatment2.png}
				\\ \tiny{DiD with variations in treatment timing. Image source: \cite{Goodman-Bacon2021}, as reproduced by \href{https://andrewcbaker.netlify.app/2019/09/25/difference-in-differences-methodology/}{\underline{\smash{A.C. Baker}}}}
			\end{center}
		\end{frame}

		\begin{frame}{notation}
			Turns out: 2WFE DiD estimator is a weighted average of these comparisons
			\begin{center}
				\includegraphics[width=.575\textwidth]{GB_staggeredDiff.jpg}
				\\ \tiny{Bacon decomposition theorem. Image source: \cite{Goodman-Bacon2021}, see also his  \href{https://twitter.com/agoodmanbacon/status/1039126592604303360}{\underline{\smash{tweet-thread}}}}
			\end{center}
		\end{frame}

			\begin{frame}{notation}
				Even worse, a positive but staggered slope change, can even change the sign of the DD estimate. \\
				\centering 
					\animategraphics[width=.7\textwidth,controls]{10}{GoodmanBacon-}{0}{531}
					\newline \tiny{Staggered treatment with linear slope change. Image source: \cite{Goodman-Bacon2021}, see also his  \href{https://twitter.com/agoodmanbacon/status/1039126592604303360}{\underline{\smash{tweet-thread}}}}
			\end{frame}

	\subsection{solutions}
		\begin{frame}{Proposed solutions I}
			Several works on the issue:
			\\ \vspace*{.25cm}
			\begin{itemize}
				\item<2-> \cite{Athey2020}
					\begin{itemize}
						\item under random adoption dates, a weighted DID is unbiased
						\item comparing treated with not-yet treated
					\end{itemize}
				\item<3-> \cite{Goodman-Bacon2021}
					\begin{itemize}
						\item a time-in-treatment weighted DiD, fixing the weights to gain balance
						\item R package \href{https://cran.r-project.org/web/packages/bacondecomp/index.html}{\underline{\smash{bacondecomp}}}
					\end{itemize}
			\end{itemize}
		\end{frame}

		\begin{frame}{Proposed solutions II}
			Several works on the issue:
			\\ \vspace*{.25cm}
			\begin{itemize}
				\item<2-> \cite{Callaway2020}
					\begin{itemize}
						\item bootstrapped inference with pre-intervention conditioning on co-variates
						\item R package \href{https://cran.r-project.org/web/packages/did/vignettes/did-basics.html}{\underline{\smash{did}}}
					\end{itemize}
				\item<3-> \cite{Sun2020a}
					\begin{itemize}
						\item time-to-treatment (cohort) weighted approach, comparing to never-treated
						\item implemented in \href{https://cran.r-project.org/web/packages/fixest/vignettes/fixest_walkthrough.html}{\underline{\smash{fixest}}} R package
					\end{itemize}
			\end{itemize}
		\end{frame}

		\begin{frame}{Heterogeneous Treatment}
			Further works on DiD design / application issues:
			\\ \vspace*{.25cm}
			\begin{itemize}
				\item \cite{DeChaisemartin2018}
					\begin{itemize}
						\item allowing for treatment effects "heterogeneous across groups and over time periods"
						\item R package \href{https://cran.r-project.org/web/packages/DIDmultiplegt/index.html}{\underline{\smash{DIDmultiplegt}}}
					\end{itemize}
				\item \cite{Imai2020}
					\begin{itemize}
						\item identify some problematic negative weigthing in 2WFE
						\item they propose weighting or matching \href{https://imai.fas.harvard.edu/research/twoway.html}{\underline{\smash{Imai 2020}}}
					\end{itemize}
			\end{itemize}
		\end{frame}

		\begin{frame}{Non-parallel trends}
			However, all these approaches, assume (to some extent) parallel trends
			\\ \vspace*{.25cm}
			\begin{itemize}
				\item Jonathan Roth \citeyear{Roth2021}
					\begin{itemize}
						\item test for parallel-trends, R package \href{https://github.com/jonathandroth/pretrends}{\underline{\smash{pretends}}}
						\item robust inference strategy under non-parralel trends \cite{Rambachan2020}, R package \href{https://github.com/asheshrambachan/HonestDiD}{\underline{\smash{HonestDiD}}}
					\end{itemize}
				\item \cite{Wooldridge2021}
					\begin{itemize}
						\item Two-Way Mundlak Regression (cohort-weighted TWFE)
						\item standard strategies for heterogeneous trends available and faciliate nonlinear models
					\end{itemize}
			\end{itemize}
		\end{frame}

		\begin{frame}{Summary}
			\begin{center}
				\includegraphics[width=.7\textwidth]{DIDsummary.png}
				\captionof{figure}{\href{https://twitter.com/davidfromterra/status/1390649644418584579?s=20}{\underline{\smash{Schönholzer 2022}}}, cf. \cite{Roth2021a}}
			\end{center}
		\end{frame}

		\begin{frame}{Summary}
			Re: DiD and (adjusted) 2WFE estimators \\ \vspace*{.5cm}
			\begin{itemize}
				\item compare treated individual (units) with a reference group
				\item who we compare to whom deserves special attention
				\item software solutions available
			\end{itemize}
		\end{frame}

% ------------------------------------------------------------------------------
	\begin{frame}[t,allowframebreaks]{References}
		% [t,allowframebreaks]
	  \printbibliography
	\end{frame}
% ------------------------------------------------------------------------------

\end{document}
