\documentclass[aspectratio=169]{beamer}
%\documentclass{beamer}
%%%CHOOSE ASPECT RATIO ABOVE%%%

\usetheme{LU}

\usepackage[utf8]{inputenc}
\usepackage{csquotes}
\usepackage[british]{babel}
\usepackage{graphicx}
% \usepackage{booktabs}
\usepackage{makecell}
\usepackage{textcomp}
\usepackage{listings}

\renewcommand\theadfont{\tiny}
\renewcommand*{\bibfont}{\scriptsize}

% ------------------------------------------------------------------------------

\title[Causal Inference]{Matching and synthetic controls \newline}

\titlecolor{LUIvory} % Choose between LUPink, LULBlue, LUIvory, LUGreen
\titleimage{\includegraphics[scale=.955]{Grayscale-Globe.jpg}}
\author{Nils Droste}
\subtitle{}
\date{dd mm yyyy}
\institute{Lund University\\Department for Political Science}
\newcommand{\conference}{2021 ClimBEco course}

% bibliography
\addbibresource{E:/Dropbox/Dokumente/references/library.bib}

\begin{document}

% ------------------------------------------------------------------------------

\titleframe


% ------------------------------------------------------------------------------
\section{Introduction}

	\begin{frame}{Causal Inference from observational data}
		\textbf{\textit{Synopsis}}: Today, we will be looking into methods that help us find (aka \textit{match}) or simulate (aka \textit{synthesize}) a control group for  inferring causal effects from observational data, and its recent developments \\ \vspace*{.25cm}
		In particular, we will develop an understanding of\\ \vspace*{.25cm}
		\begin{itemize}
			\item<2-> matching approaches
			\item<3-> synthetic controls
			\item<4-> machine-based learning methods
		\end{itemize}
	\end{frame}

	\begin{frame}{Intuition}
		Consider a situation where the untreated are very different from the treated:
		\\ \vspace*{.05cm}
		\begin{center}
			\includegraphics[width=\textwidth]{Matching.jpg}
			\\\tiny{ Image source: \cite{Schleicher2020}}
		\end{center}
	\end{frame}

	\begin{frame}{basic conditions}
		The classical overarching conditions for robust causal inference:
		\\ \vspace*{.25cm}
		\begin{itemize}
			\item stable unit treatment value assumption (SUTVA)
			\begin{itemize}
				\item treating one individual unit does not affect another's (potential) outcome
				\item treatment is comparable [no (strong) variation in treatment]
			\end{itemize}
			\item<2-> unconfoundedness (strong ignorability)
			\begin{itemize}
				\item<2-> $(Y(1), Y(0)) \perp T$: treatment assignment is independent of the outcomes
				\item<2-> i.e. no omitted variable bias (recall the storch example)
				\item<2-> or, at least, conditional unconfoundedness  $(Y(1), Y(0)) \perp T {}|{} X$
			\end{itemize}
		\end{itemize}
		\vspace*{.25cm} \onslide<3-> {$\rightarrow \pi(X_i) = Pr(D_i = 1 | X_i)$ or \textit{propensity score} can be used for matching}
		\\ \onslide<4> {$\rightarrow$ but should maybe not (\cite{King2019}), we will see alternatives }
	\end{frame}
% ------------------------------------------------------------------------------
\section{Matching}

	\begin{frame}{Overview}
		Here is a general overview of possible matching methods
		\\ \vspace*{.05cm}
		\begin{center}
			\includegraphics[width=\textwidth]{MatchingMethods.png}
			\\\tiny{ Image source: \href{https://humboldt-wi.github.io/blog/research/applied_predictive_modeling_19/matching_methods/}{\underline{\smash{Sizemore and Alkurdi 2019}}}}
		\end{center}
	\end{frame}

% ------------------------------------------------------------------------------
\section{Synthetic Controls}


% ------------------------------------------------------------------------------
	\begin{frame}[t,allowframebreaks]{References}
		% [t,allowframebreaks]
	  \printbibliography
	\end{frame}
% ------------------------------------------------------------------------------

\end{document}
