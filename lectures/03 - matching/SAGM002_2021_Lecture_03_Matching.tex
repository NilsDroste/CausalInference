\documentclass[aspectratio=169]{beamer}
%\documentclass{beamer}
%%%CHOOSE ASPECT RATIO ABOVE%%%

\usetheme{LU}

\usepackage[utf8]{inputenc}
\usepackage{csquotes}
\usepackage[british]{babel}
\usepackage{graphicx}
% \usepackage{booktabs}
\usepackage{makecell}
\usepackage{textcomp}
\usepackage{listings}

\renewcommand\theadfont{\tiny}
\renewcommand*{\bibfont}{\scriptsize}

% ------------------------------------------------------------------------------

\title[Causal Inference]{Matching and synthetic controls \newline}

\titlecolor{LUIvory} % Choose between LUPink, LULBlue, LUIvory, LUGreen
\titleimage{\includegraphics[scale=.955]{Grayscale-Globe.jpg}}
\author{Nils Droste}
\subtitle{}
\date{dd mm yyyy}
\institute{Lund University\\Department for Political Science}
\newcommand{\conference}{2021 ClimBEco course}

% bibliography
\addbibresource{E:/Dropbox/Dokumente/references/library.bib}

\begin{document}

% ------------------------------------------------------------------------------

\titleframe


% ------------------------------------------------------------------------------
\section{Introduction}

		\begin{frame}{Causal Inference from observational data}
			\textbf{\textit{Synopsis}}: Today, we will be looking into methods that help us find (aka \textit{match}) or simulate (aka \textit{synthesize}) a control group for  inferring causal effects from observational data, and its recent developments \\ \vspace*{.25cm}
			In particular, we will develop an understanding of\\ \vspace*{.25cm}
			\begin{itemize}
				\item<2-> matching approaches
				\onslide<3->{
					\begin{itemize}
						\item classical
						\item machine-based learning
					\end{itemize}
					}
				\item<4> synthetic controls

			\end{itemize}
		\end{frame}

		\begin{frame}{Intuition}
			Consider a situation where the untreated are very different from the treated:
			\\ \vspace*{.05cm}
			\begin{center}
				\includegraphics[width=\textwidth]{Matching.jpg}
				\captionof{figure}{ Image source: \cite{Schleicher2020}}
			\end{center}
		\end{frame}

		\begin{frame}{Intuition}
			Consider a situation where the untreated are very different from the treated:
			\\ \vspace*{.05cm}
			\begin{center}
				\includegraphics[width=\textwidth]{MatchingDef.jpg}
				\captionof{figure}{ Image source: Image source: \href{https://humboldt-wi.github.io/blog/research/applied_predictive_modeling_19/matching_methods/}{\underline{\smash{Sizemore and Alkurdi 2019}}}}
			\end{center}
			\onslide<2> {$\rightarrow$ \underline{matching is a \textbf{\textit{pre-analytical procedure}}, allowing unbiased inference.}}
		\end{frame}

		\begin{frame}{Procedure}
			\begin{center}
				\only<1>{\includegraphics[width=.7\textwidth]{MatchingProcedure1.jpg}}
				\onslide<2>{\includegraphics[width=.7\textwidth]{MatchingProcedure2.jpg}}
				\captionof{figure}{ Image source: \cite{Schleicher2020}}
			\end{center}
		\end{frame}

		\begin{frame}{Basic conditions}
			The classical overarching conditions for robust causal inference:
			\\ \vspace*{.25cm}
			\begin{itemize}
				\item stable unit treatment value assumption (SUTVA)
				\begin{itemize}
					\item treating one individual unit does not affect another's (potential) outcome
					\item treatment is comparable [no (strong) variation in treatment]
				\end{itemize}
				\item<2-> unconfoundedness (strong ignorability)
				\begin{itemize}
					\item<2-> $(Y(1), Y(0)) \perp D$: treatment assignment is independent of the outcomes
					\item<2-> i.e. no omitted variable bias (recall the storch example)
					\item<2-> or, at least, conditional unconfoundedness  $(Y(1), Y(0)) \perp D {}|{} X$
				\end{itemize}
			\end{itemize}
			\vspace*{.25cm} \onslide<3-> {$\rightarrow \pi(X_i) = Pr(D_i = 1 | X_i)$ or \textit{propensity score} can be used for matching}
			\\ \onslide<4> {$\rightarrow$ but should maybe not (\cite{King2019}), we will see alternatives }
		\end{frame}

% ------------------------------------------------------------------------------
\section{Matching}

		\begin{frame}{Overview}
			Here is a general overview of possible matching methods
			\\ \vspace*{.05cm}
			\begin{center}
				\includegraphics[width=\textwidth]{MatchingMethods.png}
				\captionof{figure}{ Image source: \href{https://humboldt-wi.github.io/blog/research/applied_predictive_modeling_19/matching_methods/}{\underline{\smash{Sizemore and Alkurdi 2019}}}}
			\end{center}
		\end{frame}

		\begin{frame}{Notation}
			Consider that we aim to estimate \textit{conditional average treatment effect} (CATE) (cf. \cite{Abrevaya2015})
			\begin{equation}
				CATE = E(Y(1)-Y(0)|X=x)
			\end{equation}
			\only<2>{How to find the sufficiently similar subsamples?}
			\onslide<3->{
				\hspace*{-.3cm} King and Nielsen (\citeyear{King2019}) formulate a general pruning (\textit{matching}) function $M$:
				\begin{equation}
					X_\ell = M(X|A_\ell, T_i = 1, T_j = 0, \delta) \equiv M(X|A_\ell) \subseteq X
				\end{equation}
				\\ \vspace*{.25cm}
				providing $X_\ell$, subset of matched observation based on condition $A_\ell$.
				}
			\\ \vspace*{.25cm}
			\onslide<4> {
				$\rightarrow$ in what follows we will look at different pruning method $\ell$ \\
				\hspace*{.3cm} to produce the best matched subset $\delta$.
				}
		\end{frame}

	\subsection{exact match}
		\begin{frame}{Exact matching}
			For exact matching we find exactly equal pairs
			\\ \vspace*{.5cm}
			\begin{equation}
				X_{EM} = M(X|X_i = X_j)
			\end{equation}
			\\ \vspace*{.5cm}
			\textit{Note:} $X$ can be a vector of covariates.
		\end{frame}

		\begin{frame}{Coarsened Exact Matching (CEM)}
			For coarsened exact matching we approximate
			\\ \vspace*{.5cm}
			\begin{equation}
				X_{CEM} = M(X|C_\delta(X_i) = C_\delta(X_i))
			\end{equation}
			\\ \vspace*{.5cm}
			where $C_\delta$ is a vector of same dimensions as $X$, but coarsened values, \\
			e.g. at "\textit{natural breakpoints}" such as years in one school type, levels of income, etc.
		\end{frame}

	\subsection{distance match}
		\begin{frame}{Mahalanobis Distance Method (MDM)}
			For multidimensional data, we can identify nearest neighbours in an n-dimensional space.
			\begin{center}
				\includegraphics[width=\textwidth]{MalahanobisDistance.png}
				\captionof{figure}{ Image source: \href{https://humboldt-wi.github.io/blog/research/applied_predictive_modeling_19/matching_methods/}{\underline{\smash{Sizemore and Alkurdi 2019}}}}
			\end{center}
		\end{frame}

		\begin{frame}{Propensity score matching (PSM)}
			Else, we can estimate probability of being treated, aka propensity score \\
			 $\pi(X_i) = Pr(D_i = 1 | X_i)$ by logistic regression
			\begin{center}
				\includegraphics[trim={0 0 0 4.5cm},clip,width=\textwidth]{PropScore.png}
				\captionof{figure}{ Image source: \href{https://humboldt-wi.github.io/blog/research/applied_predictive_modeling_19/matching_methods/}{\underline{\smash{Sizemore and Alkurdi 2019}}}}
			\end{center}
		\end{frame}

		\begin{frame}[fragile]{example}
				\begin{verbatim}
				library(tidyverse)
				library(MatchIt)

				data("lalonde")
				lalonde <- lalonde %>% as_tibble()

				m.out <- matchit(treat ~ age + educ + race + married +
				                 nodegree + re74 + re75, data = lalonde,
				                 method = "full")
			\end{verbatim}
		\end{frame}

		\begin{frame}[fragile]{example}
			\begin{verbatim}
				> m.out
				A matchit object
				 - method: Optimal full matching
				 - distance: Propensity score
				             - estimated with logistic regression
				 - number of obs.: 614 (original), 614 (matched)
				 - target estimand: ATT
				 - covariates: age, educ, race, married, nodegree, re74, re75
 	 		\end{verbatim}
		\end{frame}

		\begin{frame}[fragile]{example}
			\begin{verbatim}
					plot(m.out, type = "ecdf", which.xs = c("age", "re74", "married"))
			\end{verbatim}
			\includegraphics[width=.85\textwidth]{eCDF.png}
				\begin{center}
				\tiny{ Code source: \href{https://kosukeimai.github.io/MatchIt/articles/assessing-balance.html}{\underline{\smash{Greifer 2020}}}}
			\end{center}
		\end{frame}

		\begin{frame}[fragile]{example}
			\begin{verbatim}
				psFormula <- formula(treat ~ age + educ + race
				                     + married + nodegree + re74 + re75)

				lalonde$p.score <-
				  glm(psFormula, data = lalonde,
				      family = "binomial")$fitted.values

				lalonde$att.weights <-
				  with(lalonde, treat + (1-treat)*p.score/(1-p.score))
			\end{verbatim}
		\end{frame}

		\begin{frame}[fragile]{example}
			\begin{verbatim}
				bal.plot(f.build("treat", covs0),
				         data = lalonde, var.name = "p.score",
				         weights = "att.weights", distance = "p.score",
				         method = "weighting", which = "both")
			\end{verbatim}
			\begin{center}
				\includegraphics[width=\textwidth]{pscore.png}
				\tiny{Code source: \href{https://mran.microsoft.com/snapshot/2017-08-06/web/packages/cobalt/vignettes/cobalt_basic_use.html}{\underline{\smash{Greifer 2020}}}}
			\end{center}
		\end{frame}

		\begin{frame}{Intermediate discussion}
			There is a bit of critique on PSM
			\begin{itemize}
				\item King and Nielsen (\citeyear{King2019})
				\begin{itemize}
					\item \textit{"PSM is ... uniquely blind to the often large portion of imbalance"}
					\item \textit{"easy to avoid by switching to one of the other popular methods of matching"}
					\item i.e.: CEM and MDM
				\end{itemize}
				\item Sizemore and Alkurdi (\citeyear{Sizemore2019})
				\begin{itemize}
					\item test PSM against machine learning based methods
					\item logistic PSM $\succ$ random forest PSM  $\succ$ genetic matching
					\item CEM ???
				\end{itemize}
			\end{itemize}
		\end{frame}

	\subsection{machine-learning}
		\begin{frame}{Random forest (RF)}
			RF are multiple regression trees classifying the data by partitioning
			\begin{center}
				\includegraphics[trim={0 .75cm 0 1.5cm}, clip, width=.7\textwidth]{RandomForest.png}
				\captionof{figure}{Code source: \href{https://en.wikipedia.org/wiki/Random_forest}{\underline{\smash{Wikipedia}}}}
			\end{center}
			We can use this to predict treatment (aka propensity scores)
		\end{frame}

		\begin{frame}{eXtreme Gradient Boosting (XGBoost)}
			Machine learning such as XGBoost or even ensembles can also be used to
			\begin{center}
				\includegraphics[width=.9\textwidth]{XGBoost.png}
				\captionof{figure}{Code source: \href{https://blog.quantinsti.com/xgboost-python/}{\underline{\smash{Quant Insti}}}}
			\end{center}
			$\rightarrow$ predict treatment (aka propensity scores)
		\end{frame}

		\begin{frame}{Genetic matching}
			Genetic Matching combines PSM and MDM
			\begin{equation}
				GMD(X_i, X_j, W) = \sqrt{(X_i)^T(S^{-\frac{1}{2}})^TWS^{-\frac{1}{2}}(X_i-X_j)}
			\end{equation}
			\begin{center}
				\includegraphics[width=.3\textwidth]{GenMatch.png}
					\captionof{figure}{ Image source: \href{https://humboldt-wi.github.io/blog/research/applied_predictive_modeling_19/matching_methods/}{\underline{\smash{Sizemore and Alkurdi 2019}}}}
			\end{center}
		\end{frame}

	\subsection{model comparison}
		\begin{frame}{comparison - fitting distributions}
			\begin{center}
				\includegraphics[width=.9\textwidth]{comparison.png}
				\captionof{figure}{plotting model comparisons for coviarates of the lalonde data set}
			\end{center}
		\end{frame}

		\begin{frame}{comparison - mean absolute error}
			\begin{center}
				\includegraphics[width=\textwidth]{Modelcomparison.png}
				\captionof{figure}{plotting model comparisons for the lalonde data set, cf. \cite{Colson2016}}
			\end{center}
		\end{frame}

		\begin{frame}{comparison - summary}
			\begin{itemize}
				\item<1-> for the comparison above I used nearest neighbour matching, reducing sample size
				\item<2-> maximizing post-match balance does not necessarily improve explanatory model power (\cite{Colson2016})
				\item<3-> possibly both sample size and balance need to be taken into account (\cite{King2017})
				\item<4-> latest approaches include almost exact matching (\cite{Dieng2018,Dieng2021}), text matching (\cite{Roberts2020a}), generalized optimal matching (\cite{Kallus2020})
				\item<5-> R packages include \href{https://cran.r-project.org/web/packages/MatchIt/vignettes/MatchIt.html}{\underline{\smash{MatchIt}}}, \href{http://sekhon.berkeley.edu/matching/}{\underline{\smash{Matching}}}, and \href{https://github.com/insongkim/PanelMatch}{\underline{\smash{PanelMatch}}}
				\item<6> for the debate around propensity score matching (\cite{King2019}), see also \href{https://p-hunermund.com/2019/02/06/why-so-much-hate-against-propensity-score-matching/amp/)}{\underline{\smash{Hünermund, (2019)}}}
			\end{itemize}
		\end{frame}

	\subsection{example}
		\begin{frame}{an example}
			Ferraro and Hanauer (\citeyear{Ferraro2014}) use matching approach (MDM) to assess the effect of protected areas on poverty reduction
			\begin{center}
				\includegraphics[trim={0 0 0 3.5cm}, clip, width=.6\textwidth]{Hanauer.jpg}
				\captionof{figure}{Causal model of PA on poverty effects, source: \cite{Ferraro2014}}
			\end{center}
		\end{frame}

% ------------------------------------------------------------------------------
\section{Synthetic Controls}

	\subsection{intuition}

		\begin{frame}{Synthetic Controls}
			What if we do only have \textit{one} treated unit?
			\begin{center}
				\includegraphics[width=.5\textwidth]{SynthControl.jpg}
				\captionof{figure}{California introduces tobacco control in 1988, cf. \cite{Abadie2010}}
			\end{center}
		\end{frame}

		\begin{frame}{a case and an idea}
			How about we compare to a weighted average of untreated?
			\begin{center}
				\includegraphics[width=.5\textwidth]{SynthControl2.jpg}
				\captionof{figure}{California introduces tobacco control in 1988, cf. \cite{Abadie2010}}
			\end{center}
		\end{frame}

		\begin{frame}{and a notation}
			\begin{equation}
				\hat Y_{t,post}(0)  = \mu + \sum_{i=1}^{N}{w_iY_{i,T}^{obs}}
			\end{equation}
			\\ \vspace*{1.5cm}
			"\textit{In other words, the imputed control outcome for the treated unit is a linear combination of the control units, with intercept $\mu$ and weights $w_i$ for control unit $i$.}" (\cite{Doudchenko2020}: 7)
		\end{frame}

		\begin{frame}{the process}
			We compare the treated to the non-treated
			\begin{center}
				\includegraphics[width=.5\textwidth]{SynthControl3.jpg}
				\captionof{figure}{A noisy control group, cf. \cite{Abadie2010}}
			\end{center}
		\end{frame}

		\begin{frame}{the process}
			and compute the difference to a counterfactual weighted set of untreated
			\begin{center}
				\includegraphics[width=.5\textwidth]{SynthControl4.jpg}
				\captionof{figure}{California vs SynthCal, cf. \cite{Abadie2010}}
			\end{center}
		\end{frame}

	\subsection{estimation}
		\begin{frame}{estimation}
			Recall the ordinary least square estimate (OLS)
		\begin{center}
			\includegraphics[trim={6.5cm 5.25cm 5cm 5cm}, clip, width=.55\textwidth]{OLS.jpg}
				\captionof{figure}{OLS, img source: \href{https://serokell.io/blog/regression-analysis-overview}{\underline{\smash{Gavrilova, 2020}}}}
		\end{center}
		\end{frame}

		\begin{frame}{estimation}
			For
			\begin{math}
				\hat Y_{t,post}(0)  = \mu + \sum_{i=1}^{N}{w_iY_{i,T}^{obs}}
			\end{math}
			\\ \vspace*{.5cm}
			$\mu$ and $w_i$ can, in principle, be estimate with OLS (cf. \cite{Doudchenko2020})
			\begin{equation}
				(\hat{\mu}^{ols}, \hat{w}^{ols}) = \arg\min_{\mu, w} \sum_{s=1}^{T_0}\left({Y_{0,T_0-s+1}^{obs} - \mu - \sum_{i=1}^{N}{w_i} \cdot Y_{0,T_0-s+1}^{obs} }\right)^2
			\end{equation}
			\\ \vspace*{.25cm}
			\onslide<2> Abadie et al. \citeyear{Abadie2010} impose conditions, $\mu= 0$, $\sum_{i=1}^{N}{w_i}=1$, and $w_i \geq 0{ }\forall i$.
		\end{frame}

		\begin{frame}{estimation}
			For covariate vector {X} we would want to minimize (cf. \cite{Doudchenko2020})
			\\ \vspace*{.5cm}
			\begin{equation}
				\| Y_{t,pre}^{obs} - \mu - w^T Y_{c,pre}^{obs} \|_2^2 = \left( Y_{t,pre}^{obs} - \mu - w^T Y_{c,pre}^{obs} \right)^T \left( Y_{t,pre}^{obs} - \mu - w^T Y_{c,pre}^{obs} \right)
			\end{equation}
			\\ \vspace*{.25cm}
			This mathing is often performed on lagged outcomes $Y_{t-(1,...,T)}$ and other covariates. \onslide<2->{So, in simpler terms, $	\| X_{treat} - X_{control} W \|$  which resembles a balancing approach (á la matching).}
			\\ \vspace*{.25cm}
			\onslide<2> See Doudchenko and Imbens (\citeyear{Doudchenko2020}) for a balanced, cross-validated, elastic net type penalty approach, combining Lasso and ridge regressions to regularize $w$.
		\end{frame}

	\subsection{current development}
		\begin{frame}{current development}
			Arkhangelsky et al. \citeyear{Arkhangelsky2019} suggest a synthetic diff-in-diff approach, where \\ \vspace*{.15cm}
			SynthControl:
			\begin{equation}
				(\hat{\mu}, \hat{\beta}, \hat{\tau}^{sc}) = \arg\min_{\mu, \beta, \tau}  \sum_{1=1}^{N} \sum_{1=1}{T} \left(Y_{it} - \mu - \beta_t - W_{it}\tau \right)^2 \hat{w}_i^{SC}
			\end{equation}
			DiD:
			\begin{equation}
				(\hat{\mu}, \hat{\alpha}, \hat{\beta}, \hat{\tau}^{did}) = \arg\min_{\mu, \alpha, \beta, \tau}  \sum_{1=1}^{N} \sum_{1=1}{T} \left(Y_{it} - \mu - \alpha_i -\beta_t - W_{it}\tau \right)^2
			\end{equation}
			SynthDiD:
			\begin{equation}
				(\hat{\mu}, \hat{\alpha}, \hat{\beta}, \hat{\tau}^{sdid}) = \arg\min_{\mu, \beta, \tau}  \sum_{1=1}^{N} \sum_{1=1}{T} \left(Y_{it} - \mu - \alpha_i -\beta_t - W_{it}\tau \right)^2 \hat{w}_i \hat{\lambda}_t
			\end{equation}
		\end{frame}

		\begin{frame}{intermediate summary}
			A synthetic control approach allows us to \\ \vspace*{.5cm}
			\begin{itemize}
				\item compare a single treated unit group with an untreated quasi-counterfactual
				\item you can compute placebo tests for the effect on an untreated unit
				\item so far, has not been \text{widely} applied (for examples see \cite{Abadie2020})
				\item I think it is so far underestimated (i.e. by applied researchers)
			\end{itemize}
		\end{frame}

	\subsection{example}
		\begin{frame}{software}
			available packages
			\begin{itemize}
				\item \href{https://cran.r-project.org/web/packages/Synth/index.html}{\underline{\smash{Synth}}}
				\item \href{https://github.com/synth-inference/synthdid}{\underline{\smash{synthdid}}}
				\item \href{https://github.com/hollina/scul/}{\underline{\smash{scul}}}
				\item \href{https://yiqingxu.org/software/gsynth/gsynth_examples.html}{\underline{\smash{gsynth}}}
			\end{itemize}
		\end{frame}

		\begin{frame}{an example}
			Bayer and Aklin (\citeyear{Bayer2020}) use synthetic controls to assess the effect of EU Emission Trading System (ETS) on CO$_2$ emissions
			\begin{center}
				\only<1>{\begin{center}\includegraphics[width=.5\textwidth]{Bayer1.jpg} \captionof{figure}{Effect of the EU ETS over time, source: \cite{Bayer2020}}\end{center}}
				\onslide<2>{\includegraphics[width=.5\textwidth]{Bayer2.jpg}}
				\captionof{figure}{Effect of the EU ETS over time, source: \cite{Bayer2020}}
			\end{center}
		\end{frame}

% ------------------------------------------------------------------------------
	\begin{frame}[t, allowframebreaks]{References}
		% [t,allowframebreaks]
	  \printbibliography
	\end{frame}
% ------------------------------------------------------------------------------

\end{document}
