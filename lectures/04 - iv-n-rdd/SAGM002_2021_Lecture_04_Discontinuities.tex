\documentclass[aspectratio=169]{beamer}
%\documentclass{beamer}
%%%CHOOSE ASPECT RATIO ABOVE%%%

\usetheme{LU}

\usepackage[utf8]{inputenc}
\usepackage{csquotes}
\usepackage[british]{babel}
\usepackage{graphicx}
% \usepackage{booktabs}
\usepackage{makecell}
\usepackage{textcomp}
\usepackage{listings}
\usepackage{multirow}

\renewcommand\theadfont{\tiny}
\renewcommand*{\bibfont}{\scriptsize}
\newcommand{\Var}{\operatorname{Var}}
\newcommand{\Cov}{\operatorname{Cov}}

% ------------------------------------------------------------------------------

\title[Causal Inference]{Instrumental Variables and \newline regression discontinuity designs}

\titlecolor{LUIvory} % Choose between LUPink, LULBlue, LUIvory, LUGreen
\titleimage{\includegraphics[scale=.955]{Grayscale-Globe.jpg}}
\author{Nils Droste}
\subtitle{}
\date{dd mm yyyy}
\institute{Lund University\\Department for Political Science}
\newcommand{\conference}{2021 ClimBEco course}

% bibliography
\addbibresource{E:/Dropbox/Dokumente/references/library.bib}

\begin{document}

% ------------------------------------------------------------------------------

\titleframe


% ------------------------------------------------------------------------------
\section{Introduction}


% ------------------------------------------------------------------------------
\section{Instrumental Variables}
	\subsection{intuition}
		\begin{frame}{Instrumental Variables}
			Instrumental Variable (IV)
			\begin{center}
				\includegraphics[scale=.7]{dag-iv}
				\captionof{figure}{Using exogeneous variation in instrument to close back-door. Image source: \href{http://nickchk.com/causalgraphs.html}{\underline{\smash{Huntington-Klein 2018}}}}
			\end{center}
		\end{frame}

		\begin{frame}{Instrumental Variables}
			An exemplary study
			\begin{center}
				\includegraphics[scale=.4]{Larreboure}
				\captionof{figure}{\cite{Larreboure2021}}
			\end{center}
		\end{frame}

		\begin{frame}{Instrumental Variables}
			But there may be plenty of causal pathways in reality
			\begin{columns}
				\begin{column}{0.5\textwidth}
					\includegraphics[scale=.175]{WeatherIV.jpg}
				\end{column}
				\begin{column}{0.5\textwidth}
					\includegraphics[scale=.25]{WeatherIV2.jpg}
				\end{column}
			\end{columns}
			\begin{center}
				\scriptsize{Weather IV and (in)dep vars, in general (left) and with temporal variations (right). Source: \cite{Mellon2020}}
			\end{center}
		\end{frame}

	\subsection{formalities}
		\begin{frame}{Instrumental Variables}
			\begin{columns}
				\begin{column}{0.66\textwidth}
					\begin{center}
			 			\animategraphics[width=.7\textwidth,controls]{10}{IV-}{0}{174}
					\end{center}
				\end{column}
				\begin{column}{0.33\textwidth}
					Two-Stage Least Square (2SLS) estimator \\ \vspace*{.2cm}
					\scriptsize{1. stage: regress $Z$ on $X$:}
					\normalsize{\begin{equation}
						X_{i} = \alpha_1 + \beta_1 Z_{i} + \gamma_1 W_{i} + \varepsilon_{1,i}
					\end{equation}} \\
					\scriptsize{and predict the variation in $X$ explained by $Z$: $\widehat{X} = \beta_1 Z_{i}$. \\ \vspace*{.2cm}
					2. stage: plug in $\widehat{X}$ to estimate the variation in $Y$ not explained by confounder $W$: \\ \vspace*{-.4cm}}
					\normalsize{\begin{equation}
						Y_{i} = \alpha_2 + \beta_2 \widehat{X}_{i} + \gamma_2 W_{i} + \varepsilon_{2,i}
					\end{equation}}
				\end{column}
			\end{columns}
		\end{frame}

		\begin{frame}{conditions}
			There are important conditions to consider
			\begin{itemize}
				\item<1-> \textit{relevance} of instrument for predicting $Y$ $\rightarrow E((\widehat{X}_i|Z=1)-(\widehat{X}_i|Z=0)) \neq 0$, aka $Z$ is correlated with $X$, and thus with $Y$.
				\item<2-> \textit{exclusion} restriction of $Z$ being independent of $Y$: $E(\epsilon_i, Z_i | W_i) = 0$, aka no backdoor $Z \rightarrow Y$ or endogeneity, i.e. no relation with omitted variables.
			\end{itemize}
			\vspace*{.5cm}
			\onslide<3->{Now, let us see how to formulate this in the potential outcome notation.} \\ \vspace*{.25cm}
			\onslide<4>{For this let treatment or participation again be denoted by $D$, now as a function of the instrument $\rightarrow$ $D_i(Z_i)$, the \textit{intention to treat}.}
		\end{frame}

		\begin{frame}{notation}
			Step by step \\ \vspace*{.25cm}
			\begin{itemize}
				\item<1-> Imbens and Angrist (\citeyear{Imbens1994}) formulate local average treatment effect (LATE)
				\begin{itemize}
					\item<1-> for the \textit{subpopulation} responding to instrument $Z$, \\ that is those who participate $P(1)$ in treatment $D$
				\end{itemize}
			\end{itemize}
			\vspace*{.25cm}
			\onslide<2->{
				\begin{equation}
					E(Y_i|Z_i=1)-E(Y_i|Z_i=0)
				\end{equation}
			}
			\begin{itemize}
				\item []
				\begin{itemize}
					\item<3-> here the LATE is given by $P(1) \cdot E \lbrack Y_i(1) - Y_i(0)|D_i(1) = 1 \rbrack $
					\item<4-> as long as participation $P(1)>P(0)$ and $D_i(1) \geq D_i(0) \forall i$, aka monotonic \\ (or $\leq$, respectively)
				\end{itemize}
			\end{itemize}
			\vspace*{.25cm}
			\onslide<5>Why? % Else, we can not arrive at identifying LATE by $ E( Y_i(1)-Y_i(0) | D_i(1)-D_i(0)=1 ) $.
		\end{frame}

		\begin{frame}{Who's gonna be "treated"}
			Consider Angrist, Imbens and Rubin (\citeyear{Angrist1996a})
			\\\vspace*{.25cm}
				\begin{table}
					\begin{tabular}{|c|c|c|c|}
						\cline{3-4}
						\multicolumn{2}{c}{\multirow{2}{*}{}}                           & \multicolumn{2}{|c|}{$Z_i = 0$}     \\ \cline{3-4}
						\multicolumn{2}{c}{}                                            & \multicolumn{1}{|c|}{$D_i(0) = 0$}      &  $D_i(0) = 1$   \\ \cline{1-4}
						\multicolumn{1}{|c|}{\multirow{2}{*}{$Z_i = 1$}} & $D_i(1) = 0$ & Nevertaker         &  Defier        \\ \cline{2-4}
						\multicolumn{1}{|c|}{}                           & $D_i(1) = 1$ & Complier           &  Always-taker  \\ \cline{1-4}
					\end{tabular}
				\end{table}
				\vspace*{.15cm}
			\begin{itemize}
				\item<2->  "\textit{if people are more likely, on average, to participate given Z = w than given Z = z, then anyone who would participate given Z = z must also participate given Z = w}" (\cite{Imbens1994}) \\ \vspace*{.15cm}
				\onslide<3->{$\rightarrow$ assumes existence of only one of compliers or defiers, e.g. \textit{no one} defies}
				\item<4-> allows valid estimate of LATE, but may not always be realistic
			\end{itemize}
		\end{frame}

		\begin{frame}{Relaxing monotonicity assumption}
			de Chaisemartin (\citeyear{DeChaisemartin2017}) shows IVs can be valid without strong monotonicity \\ \vspace*{.15cm}
			\begin{itemize}
				\item "\textit{If there are defiers in the population, we only know that 2SLS estimates a weighted difference between the effect of the treatment among compliers and defiers}"
				\item<2->  a weak solution: $P(C_F)=P(F)$ and $E(Y(1) - Y(0) |  C_F) = E(Y(1) - Y(0) |  F) $
				\item<3-> "\textit{is satisfied if a subgroup of compliers accounts for the same percentage of the population as defiers and has the same LATE}"
				\item<4-> I believe this can be approached with matching, too. See Murray et al. (\citeyear{Murray2021}) who suggest to estimate the intention to treat D(Z) with logistic regression, providing leeway for a propensity score or other matching approach (cf. \cite{Hirano2003, Rosenbaum1984}).
			\end{itemize}
		\end{frame}

	\subsection{technicalities}
		\begin{frame}{Weak instruments}
			When instruments are only weakly correlated with treatment, reconsider
			\begin{equation}
				Y_{i} = \alpha_2 + \beta_2 {D}_{i} + \gamma_2 W_{i} + \varepsilon_{i}
			\end{equation}
			\begin{equation}
				X_{i} = \alpha_1 + \beta_1 Z_{i} + \gamma_1 W_{i} + \upsilon_{i}
			\end{equation}
			\vspace*{.15cm}
			A condition was relevance, i.e. $E((D_i|Z=1)-(D_i|Z=0)) \neq 0$, or $\Cov(Z_i,D_i|W_i) \neq 0$
			\begin{itemize}
			 \item to estimate IV, $\widehat{\beta}_2=\frac{\Cov(Y_i,Z_i)}{\Cov(Z_i,D_i|W_i)}$
			 \item problematic when $\Cov(Z_i,D_i|W_i) \to 0$ as $\Delta \beta_2$ grows large even for small variations
			\end{itemize}
		\end{frame}

		\begin{frame}{Weak instruments}
			A range of techniques for robust parameter estimation in weak IV 2SLS \\ \vspace*{.15cm}
			\begin{itemize}
			 \item<1-> F-test for strong enough instruments (\cite{Stock2003})
			 \item<2-> heteroskedasticity and autocorrelation robust for just identified models (\cite{Chernozhukov2008})
			 \item<3-> heteroskedasticity, autocorrelation and cluster robust in a more general setting (\cite{Montiel2013})
			 \item<4-> a more powerfull test with t-ratio critical value adjustments for significance testing (\cite{Lee2020})
		 	\end{itemize}
			\vspace*{.15cm}
			\onslide<5>{$\rightarrow$ We will look into some (basic) testing in the seminar.}
		\end{frame}

		\begin{frame}{intermediate summary}
			Instrumental variables allow us to \\ \vspace*{.5cm}
			\begin{itemize}
				\item isolate a treatment effect by looking at the outcomes of exogeneously caused variation
				\item it is considered a very robust causal inference, but assumptions are somewhat crucial
				\item mainly it is theory and reason that make a "valid instrument"
				\item there is loads of tests, I do not think they alone suffice
			\end{itemize}
		\end{frame}

% ------------------------------------------------------------------------------
\section{Regression Discontinuity}
	\subsection{intuition}
		\begin{frame}{Intuition}
			Suppose we believe there is an effect for which assignment is non-random, but the cut-off at which treatment is assigned is quasi-random (cf. \cite{Thistlewaite2016}). \\ \vspace*{.15cm}
			\begin{center}
				\onslide<2>{\includegraphics[width=.75\textwidth]{SpatialRDD_example.jpg}
				\captionof{figure}{The border between Haiti and the Dominican Republic. Image source: \cite{Wuepper2020}}
				}
			\end{center}
		\end{frame}

		\begin{frame}{concept}
			\begin{center}
				\includegraphics[width=.5\textwidth]{RDDconcept}
				\captionof{figure}{The RDD concept and the effect of bin size choice. Image source: \cite{Cattaneo2016}}
			\end{center}
		\end{frame}

		\begin{frame}{DAGs}
			Regression-Discontinuity-Design (RDD)
			\begin{columns}
				\begin{column}{0.5\textwidth}
					\begin{center}
						\onslide<1->{\includegraphics[width=\textwidth]{dag-rdd-1}}
					\end{center}
				\end{column}
				\begin{column}{0.5\textwidth}
					\begin{center}
						\onslide<2->{\includegraphics[width=\textwidth]{dag-rdd-2}}
					\end{center}
				\end{column}
			\end{columns}
			\scriptsize{Focussing on effects just around the cutoff value. Image source: \href{http://nickchk.com/causalgraphs.html}{\underline{\smash{Huntington-Klein 2018}}}}
			\onslide<2>{$rigtharrow$ Do you see the IV in RDD?}
		\end{frame}

	\subsection{notation}

		\begin{frame}{notation}
			Let us formulate in the potential outcomes notation. Suppose there is a outcome $(Y(1) ,Y(0))$ that depends on treatment $D$ and covariate $X$.
			\\ \vspace*{.15cm}
			\onslide<2->{While $Y(X)$ is assumed to be continous, treatment $D$ kicks in at a quasi-random threshold of $\overline{x}$, such that}
			\onslide<3->{
				\begin{equation}
					D_i=
					\begin{cases}
							1 \text{if}  x_i \geq \overline{x} \\
							0 \text{if}  x_i < \overline{x}
					\end{cases}
				\end{equation}
				}
			\vspace*{.25cm}
			\onslide<4->{
				The identifying assumption is again that treatment assignment is independent of outcomes $E(Y(1)-Y(0) \perp D | X = \overline{x})$.
				}
		\end{frame}

		\begin{frame}{Sharp RDD}
			The cutcoff at $\overline{x}$ can be \textbf{sharp}
				\begin{itemize}
					\item<2-> in which case there is no overlap on both sides of $\overline{x}$
					\item<3-> we assume the outcomes would have been smooth in the absencee of treatment (aka extrapolate a "bin" beyond the threshold)
				\end{itemize}
				\begin{center}
					\onslide<3->{\includegraphics[width=.5\textwidth]{SRDD.jpg} \captionof{figure}{\cite{Imbens2008}}}
				\end{center}
				\begin{itemize}
					\item<4-> and measure $\tau_{srdd} = \lim_{x \to \overline{x}} E[Y(1)|X=x] - \lim_{x \gets \overline{x}} E[Y(0)|X=x] $
					\item<5> $D$ is not just correlated but a deterministic function of $x$ (once we know $x$ and $\overline{x}$, we know $D$)
				\end{itemize}
		\end{frame}

		\begin{frame}{estimator}
			\begin{columns}
				\begin{column}{0.66\textwidth}
						\begin{center}
							\animategraphics[width=.7\textwidth,controls]{10}{RDD-}{0}{174}
						\end{center}
					\end{column}
				\begin{column}{0.33\textwidth}
					\onslide<2>{
						RDD estimation
						\begin{equation}
							Y_{i} = \alpha_i + \beta X_{it} + \gamma t_i +  \varepsilon_{it}
						\end{equation}
						\scriptsize{where $t$ indicates treatment cuttoff values $\overline{x}$: \\\vspace*{-.4cm}}
						\normalsize{\begin{equation}
							t_i=
							\begin{cases}
									1 \text{if}  x_i \geq \overline{x} \\
									0 \text{if}  x_i < \overline{x}
							\end{cases}
						\end{equation}} \\
						\scriptsize{This would often include polynomial terms to allow for non-linear functional forms (but should not, cf. \cite{Gelman2019}). Another typical approach is a local linear regression (which is displayed in the animation) or smoothing functions. }
						}
				\end{column}
			\end{columns}
		\end{frame}

		\begin{frame}{fuzzy RDD}
			Suppose the data did \textbf{not} look like this \\ \vspace*{.15cm}
			\begin{center}
				\includegraphics[width=\textwidth]{FuzzyRDD1.png}
				\captionof{figure}{
					\href{https://evalf20.classes.andrewheiss.com/example/rdd-fuzzy/}{\underline{\smash{Heiss 2020}}}
					}
			\end{center}
		\end{frame}

		\begin{frame}{fuzzy RDD}
			but rather looked like this \\ \vspace*{.15cm}
			\begin{center}
				\includegraphics[width=\textwidth]{FuzzyRDD2.png}
				\captionof{figure}{
					\href{https://evalf20.classes.andrewheiss.com/example/rdd-fuzzy/}{\underline{\smash{Heiss 2020}}}
					}
			\end{center}
		\end{frame}

		\begin{frame}{fuzzy RDD}
			So we need to evaluate \\ \vspace*{.15cm}
			\begin{center}
				\includegraphics[width=.75\textwidth]{FuzzyRDD3.png}
				\captionof{figure}{
					\href{https://evalf20.classes.andrewheiss.com/example/rdd-fuzzy/}{\underline{\smash{Heiss 2020}}}
				}
			\end{center}
			\onslide<2>{This is literally an IV setting where a different probability on two sides of the cutoff predicts participation.}
		\end{frame}

		\begin{frame}{Fuzzy RDD}
			The cutcoff at $\overline{x}$ is be \textbf{fuzzy}
				\begin{itemize}
					\item<2-> because of deniers or nevertakers etc, there is overlap on both sides of $\overline{x}$
					\item<3-> probabilities differ: $\lim_{x \to \overline{x}} Pr(Y(1)|X=x] \neq \lim_{x \gets \overline{x}} Pr(Y(0)|X=x) $
				\end{itemize}
				\begin{center}
					\onslide<3->{\includegraphics[width=.5\textwidth]{FRDD.jpg} \captionof{figure}{\cite{Imbens2008}}}
				\end{center}
				\begin{itemize}
					\item<4-> if unconfounded, $\tau_{frdd} = E[Y(1)|D=1, X=\overline{x}] -  E[Y(0)|D=1, X=\overline{x}] $
					\item<5>  which we can estimate with 2SLS, predicting $D$ in first stage, plugging estimates into second stage
				\end{itemize}
		\end{frame}


	\subsection{examples}

		\begin{frame}{examples}
			There are discontiuities in space
			\begin{center}
				\only<1>{\includegraphics[scale=.6]{SpaceRDD.jpg} \captionof{figure}{\cite{Wuepper2020b}}}
				\only<2>{\includegraphics[scale=.4]{SpaceRDD2.jpg} \captionof{figure}{Regression discontiuities in covariates but not in vegetation potential, \cite{Wuepper2020b}}}
			\end{center}
		\end{frame}

		\begin{frame}{examples}
			Time \\
			\begin{center}
				\includegraphics[scale=.4]{TimeRDD.jpg} \captionof{figure}{\cite{Hausman2018}}
			\end{center}
		\end{frame}

		\begin{frame}{examples}
			Rules \\
			\begin{center}
				\only<1>{\includegraphics[scale=.5]{AbouChadi} \captionof{figure}{\cite{AbouChadi2018}}}
				\only<2>{\includegraphics[scale=.5]{PartyCompetitionModel} \captionof{figure}{Hotelling-Downs Model of 2 Party Competition. Image Source: \href{https://blogs.warwick.ac.uk/dcstevens/entry/the_hotelling-downs_model/}{\underline{\smash{Daniel Corradi Stevens}}}}}
				\only<3>{\includegraphics[scale=.4]{AbouChadiRDD} \captionof{figure}{Mainstream party position on cultural position. Image source: \cite{AbouChadi2018}}}
			\end{center}
		\end{frame}

		\begin{frame}{examples}
			There can be kinks, aka slope shifts \\
			\begin{center}
				\includegraphics[scale=.5]{Kinks.jpg} \captionof{figure}{\cite{Ganong2018}}
			\end{center}
		\end{frame}

		\begin{frame}{examples}
			Multiple breaks \\
			\begin{center}
				\only<1>{\includegraphics[scale=.4]{Maimonides.jpg} \captionof{figure}{\cite{Angrist1999}}}
				\only<2>{\includegraphics[scale=.4]{Maimonides2.jpg} \captionof{figure}{\cite{Angrist1999}}}
			\end{center}
		\end{frame}

		\begin{frame}{software}
			available packages
			\begin{itemize}
				\item \href{https://cran.r-project.org/web/packages/rdd/}{\underline{\smash{rdd}}}
				\item \href{https://rdpackages.github.io/}{\underline{\smash{rdrobust}}}
				\item \href{https://rdpackages.github.io/}{\underline{\smash{rdlocrand}}}
				\item \href{https://rdpackages.github.io/}{\underline{\smash{rddensity}}}
				\item \href{https://rdpackages.github.io/}{\underline{\smash{rdmulti}}}
				\item \href{https://rdpackages.github.io/}{\underline{\smash{rdpower}}}
			\end{itemize}
		\end{frame}

		\begin{frame}{intermediate summary}
			Regression discontinuity designs \\ \vspace*{.5cm}
			\begin{itemize}
				\item identify a causal effect at a (quasi-)randomly occuring break point that introduces treatment
				\item are the youngest "classical" causal inference methods and seen as favorable
				\item use breaks that can occur in space, time, institutions, etc. pp.
				% \item go find some discontinuities
			\end{itemize}
		\end{frame}

% ------------------------------------------------------------------------------
	\begin{frame}[t, allowframebreaks]{References}
		% [t,allowframebreaks]
	  \printbibliography
	\end{frame}
% ------------------------------------------------------------------------------

\end{document}
