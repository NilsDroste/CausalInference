\documentclass[aspectratio=169]{beamer}
%\documentclass{beamer}
%%%CHOOSE ASPECT RATIO ABOVE%%%

\usetheme{LU}

\usepackage[utf8]{inputenc}
\usepackage{csquotes}
\usepackage[british]{babel}
\usepackage{graphicx}
% \usepackage{booktabs}
\usepackage{makecell}
\usepackage{textcomp}

\renewcommand\theadfont{\tiny}

\title[Causal Inference]{Machine learning, Bayes and \& \newline structural models }

\titlecolor{LUIvory} % Choose between LUPink, LULBlue, LUIvory, LUGreen
\titleimage{\includegraphics[scale=.955]{Grayscale-Globe.jpg}}
\author{Nils Droste}
\subtitle{e.g. authors}
\date{dd mm yyyy}
\institute{Lund University\\Department for Political Science}
\newcommand{\conference}{2021 ClimBEco course}

\renewcommand*{\bibfont}{\scriptsize}
\newcommand\prop{\mathord{\textrm{p}}}

% bibliography
\addbibresource{E:/Dropbox/Dokumente/references/library.bib}

\begin{document}

% ------------------------------------------------------------------------------

\titleframe

% ------------------------------------------------------------------------------
% \section{Outline}

	\begin{frame}{introduction}
		Today, we will talk about \\ \vspace*{.2cm}
		\begin{itemize}
			\item machine learning techniques
			\item structural equation models
			\item Bayesian inference
		\end{itemize}
		\vspace*{.2cm}
		and their relation to causal inference
	\end{frame}

	\begin{frame}{introduction}
		"There is no hierarchy in causal inference"
		\begin{center}
			\only<1>{
				% The elefant in the room \\
				\includegraphics[scale=.325]{There-is-no-hierarchy-in-causal-inference.jpg}
				\captionof{figure}{\href{https://twitter.com/MoritzPoll/status/1380514413464158211?s=20}{\underline{\smash{Moritz Poll2021}}}}
			}
			\only<2>{
				% The real elefant? \\
				\includegraphics[scale=.225]{MachineLearning.jpg}
				\captionof{figure}{\href{https://twitter.com/kareem_carr/status/1374760100079017984?s=20}{\underline{\smash{Kareem Carr 2021}}}}
			}
			\only<3>{
				% The real elefant? \\
				\includegraphics[scale=.4]{CausalHierarchy.jpg}
				\captionof{figure}{The causal hierarchy. Source: \cite{Pearl2019}}
			}
			% \only<4>{
			% 	% The real elefant? \\
			% 	\includegraphics[scale=.3275]{CausalHierarchy2.jpg}
			% 	\captionof{figure}{The causal hierarchy. Source: \cite{Bareinboim2020}}
			% }
		\end{center}
	\end{frame}

% ------------------------------------------------------------------------------
% \section{Outline}

\section{machine learning}

	\subsection{causal trees}

		\begin{frame}{causal trees}
			Recall, the conditional average treatment effect (cf. \cite{Athey2016}) \\
				\vspace*{.2cm}
			\begin{equation}
				CATE = E(Y_i(1)-Y_i(0)|X_i=x)
			\end{equation}
 			\vspace*{.2cm}
			$\rightarrow$ we could look at average differences between treated and untreated groups within partitions of the observed
		\end{frame}

		\begin{frame}{causal trees}
			We could use covariates $X_i$ to identify clusters within the observed, e.g. by recursive binary splitting\\
				\vspace*{.2cm}
				\begin{center}
					\only<1>{
						\includegraphics[scale=.5]{Forests.jpg}
						\captionof{figure}{Source: \cite{Athey2019}}
					}
					\only<2>{
						\includegraphics[scale=.4]{Forests2.jpg}
						\captionof{figure}{\href{https://www.datacamp.com/community/tutorials/decision-trees-R}{\underline{\smash{datacamp 2021}}}}
					}
					\only<3>{
						\includegraphics[scale=.4]{Forests3.png}
						\captionof{figure}{\href{https://github.com/grantmcdermott/parttree}{\underline{\smash{McDermott 2021}}}}
					}
				\end{center}
		\end{frame}

		\begin{frame}{causal trees}
			We can use classification and regression trees to estimate \\
				\vspace*{.2cm}
			\begin{equation}
				\hat{\tau}(x) = \frac{1}{|\{i:D_i  = 1, X_i \in L\}|} \sum_{\{i:D_i  = 1, X_i \in L\}}Y_i - \frac{1}{|\{i:D_i  = 0, X_i \in L\}|} \sum_{\{i:D_i  = 0, X_i \in L\}}Y_i
			\end{equation}
 			\vspace*{.2cm}
			by "recursively splitting the feature space ... into a set of leaves $L$" (\cite{Wager2018}) \\ \vspace*{.2cm}
			\onslide<2>{
				and repeat to build an ensemble of trees $\rightarrow$ random forests. This particularly well suited for heterogeneous treatment effects. Athey and Imbens (\citeyear{Athey2016}) suggest an \textit{honest} cross validation procedure for robust causal inference, see \href{https://grf-labs.github.io/grf/}{\underline{\smash{generalized random forests R package}}}.
				}
		\end{frame}

		\begin{frame}{honest crossvalidation}
		This recursive partioning uses multiple optimizations
			\begin{itemize}
				\item within sample (e.g. MSE, possibly with penalties) $\rightarrow$ danger of overfitting to sample
				\item minimizing prediction error on unseen portions of the data to reduce out-of-sample bias $\rightarrow$ (n-fold) crossvalidation
			\end{itemize}
			\vspace*{.25cm}
			\onslide<2>{
				If we do this honestly
				\begin{itemize}
					\item	we can compute average treatment effects in the splits
					\item it promises a valid CATE estimation
					\item it allows to estimate heterogenous treatment effects
				\end{itemize}
			}
		\end{frame}

		\begin{frame}{heterogenous treatment effects}
			\begin{center}
				\includegraphics[scale=.55]{RegressionTreeCATE.jpg}
				\captionof{figure}{Source: \cite{Johansson2019}}
			\end{center}
		\end{frame}

		\begin{frame}{A similar - and earlier - approach}
			Bayesian Additive Regression Trees (BART), e.g. \cite{Hill2011} \\ \vspace*{.25cm}
				\begin{itemize}
					\item estimate two models
						\begin{itemize}
							\item one on the treatment assignment process
							\item a second on the treatment effect (aka response surface)
						\end{itemize}
					\item<2-> seems to perform weel in comparison to linear regression, propensity-score matching, etc. (\cite{Hill2011})
					\onslide<3>{
						\item "\textit{BART is not constrained by prior theories}" (ibid.)
						\begin{itemize}
							\item not sure this is unproblematic
						\end{itemize}
						}
				\end{itemize}
				\vspace*{.25cm}
				\onslide<2>{For a suite of latest and different approaches read up on \cite{Carvalho2019}}
		\end{frame}

	\subsection{estimation}

		\begin{frame}{machine learning estimation}
			\begin{center}
				\includegraphics[scale=.175]{MachineLearningOptimization.jpg}
				\captionof{figure}{\href{https://towardsdatascience.com/on-optimization-of-deep-neural-networks-21de9e83e1}{\underline{\smash{Parmeet Bhatia 2020}}}}
			\end{center}
		\end{frame}

		% https://towardsdatascience.com/an-introduction-to-surrogate-optimization-intuition-illustration-case-study-and-the-code-5d9364aed51b

		\begin{frame}{estimation techniques}
			Instead of ordinary least squares \\
				\vspace*{.1cm}
			\begin{equation}
				\hat{beta}^{ols} = \arg \min_{\beta} \sum_{i=1}^{N}(Y-\beta X_i)^2
			\end{equation}
			\vspace*{.1cm}
			we can also use machine learning techniques (aka sequential updates / iterative estimation / optimization of parameter values) to estimate
			\onslide<2->{
				\begin{equation}
					\hat{beta}^{ml} = \arg \min_{\beta} \sum_{i=1}^{N}(Y-\beta X_i)^2 + \lambda (\| \beta \|_q)^{1/q}
				\end{equation}
			}
			\onslide<3>{
				with a regularizing penalty term that results in sparse models, \\
				$q=1$ for Lasso, $q=2$ ridge regression, and $q \to 0$ for best subset regression, or hybrids such as elastic nets (cf. \cite{Athey2019}) \\
				and posterior choice of $\lambda$ by (n-fold) cross-validation.
			}
		\end{frame}

		\begin{frame}{estimation techniques}
			Or we use neural networks, where we transform \\
				\vspace*{.1cm}
			\begin{equation}
				Z_{ik}^{(1)} = \sum_{j=1}^{K}\beta_{kj}{(1)}X_{ij} \quad  \textrm{ for }  \quad  k = i, ...  , K_1
			\end{equation}
				\vspace*{.1cm}
			with $K$ covariates (aka features) and latent (aka unobserved / hidden node) variable $Z_jk$, e.g. into a recitified linear function $ g(z) = z1_{z>0}	$. \\
			\onslide<2>{
				which allows us a to formulate a neural network
				\begin{columns}
					\begin{column}{0.5\textwidth}
						\begin{center}
							\href{https://www.cureus.com/articles/2260-classification-of-brain-metastases-prognostic-groups-utilizing-artificial-neural-network-approaches}{\includegraphics[scale=.2]{NeuralNet.png}}
						\end{center}
					\end{column}
					\begin{column}{0.5\textwidth}
						\begin{equation}
							Y_i = \sum_{k=1}^{K_1} \beta{k}{(2)}g \lbrack Z_{ik}^2 \rbrack + \varepsilon_i
						\end{equation}
					\end{column}
				\end{columns}
				\vspace*{.2cm}
				with a single layer $K_1$, and a non-linear transformation $g(\cdot)$ (cf. \cite{Athey2019,Farrell2021}).
			}
		\end{frame}

	\subsection{matching}
		\begin{frame}{matching or synthetic control}
			Or, we can use machine learning methods for matching or synthetic control (\cite{Doudchenko2020}) \\ \vspace*{.1cm}
			e.g to estimate weights $w$  and intercept $\mu$ for simulating counterfactual
				\vspace*{.2cm}
			\begin{equation}
				\hat Y_{t,post}(0)  = \mu + \sum_{i=1}^{N}{w_iY_{i,T}^{obs}}
			\end{equation}
			by
			\begin{align}
				(\hat{\mu}^{ols}, \hat{w}^{ols}) = \arg\min_{\mu, w} \sum_{s=1}^{T_0}\left({Y_{0,T_0-s+1}^{obs} - \mu - \sum_{i=1}^{N}{w_i} \cdot Y_{0,T_0-s+1}^{obs} }\right)^2 + \notag \\
				\underbrace{\lambda \left( \frac{1-\alpha}{2} \| w \|_2^2 + \alpha \| w \|_1 \right) }_{penalty function}
			\end{align}
		\end{frame}

		\begin{frame}{matching}
			We can also used unsupervised machine learning, such as e.g. latent dirichlet allocation, to use text as data (\cite{Gentzkow2019}) for matching a treated with an untreated population (aka matching on similar content)
			\begin{center}
				\only<1>{
					\includegraphics[scale=.4]{TextMatching.jpg}
					\captionof{figure}{Topical Inverse regression matching (TIRM). Source: \cite{Roberts2020a}}
					}
				\onslide<2>{
					\includegraphics[scale=.4]{TextMatching2.jpg}
					\captionof{figure}{Topical Inverse regression matching (TIRM). Source: \cite{Roberts2020a}}
					}
			\end{center}
		\end{frame}


% ------------------------------------------------------------------------------
\section{Bayesian inference}

		\begin{frame}{Bayes theorem}
			\begin{equation}
				 \prop(y|x)=\frac{\prop(x|y)\prop(y)}{\prop(x)}
			\end{equation}
		\end{frame}

		\begin{frame}{Bayesian inference}
			Now that we talked about sequential parameter optimization, model ensembles (aka model averaging), and posteriors, ...
			\\ \vspace*{.2cm}
			\onslide<2->{
				we are close to talk about a Bayesian approach. Suppose we formulate econometric model
				}
			\onslide<3->{
				\begin{equation}
				 	\prop(\theta|y)=\frac{\prop(y|\theta)\prop(\theta)}{\prop(y)}  \quad \varpropto \quad \prop(y|\theta)\prop(\theta)
				\end{equation}
				}
			\onslide<4>{
				with \textit{prior} parameter probability $\prop(\theta)$ and the \textit{posterior} probability (aka conditional probability) $\prop(y|\theta))$, the likelihood of observables given the parameter.
				}
		\end{frame}

		\begin{frame}{Bayesian inference}
			\onslide<1->{This framework can be used to estimate literally all types of models that we have been discussing, e.g. \\ \vspace*{.2cm}}
			\begin{itemize}
				\item<2-> Bayesian linear regression (and thus Bayesian 2SLS)
				\item<3-> Panel data (\cite{Ning2019})
				\item<4-> Regression discontinuity (\cite{Hinne2020})
				\item<4-> Ridge and Lasso regressions have their Bayesian interpretation already (cf. \cite{Athey2019})
				\item<5-> Simulating a synthetic control by Bayesian structural time series analysis (\cite{Brodersen2015})
				\item<5-> Bayesian random forests (\cite{Hill2011,Hahn2020})
			\end{itemize}
			\vspace*{.2cm}
			\onslide<5>{It has the merit of being explicit of prior beliefs about distributions (or probalities of parameters) and thus uncertainties}
		\end{frame}

% ------------------------------------------------------------------------------
\section{Structural Models}

	\subsection{why structural}

		\begin{frame}{A far greater uncertainty}
			\begin{center}
				\includegraphics[scale=.1625]{PlatosCave.jpg}
				\captionof{figure}{
					What do we really now about the world. Image source: \href{https://s.studiobinder.com/wp-content/uploads/2019/12/Platos-Allegory-of-the-Cave-Featured-Image-1.jpg}{\underline{\smash{Studio Binder 2020}}}
					}
			\end{center}
		\end{frame}

		\begin{frame}{A far greater uncertainty}
			\begin{center}
					\includegraphics[scale=.75]{BlackBoxModels.jpg}
					\captionof{figure}{	Black Box Models. Source: \cite{Zhao2021}}
			\end{center}
		\end{frame}

		\begin{frame}{We need a theory of change}
			\begin{center}
					\includegraphics[scale=.6]{SEM.jpg}
					\captionof{figure}{Directed Graph Model. Source: \href{https://plato.stanford.edu/entries/causal-models/}{\underline{\smash{Stanford Encyclopedia of Philosophy 2018}}}}
			\end{center}
		\end{frame}

		\begin{frame}{structural models}
			"[G]raphical representations of structural causal models do not require the learner - whether artificial or human - to \textit{impose any distributional or functional-form restrictions} on the underlying causal mechanisms under study. \\ \vspace*{.1cm}

			The approach remains fully nonparametric, a characteristic it shares with the potential outcomes framework.  At the same time,  however,  crucial identification assumptions,  suchas ignorability, are derived from the properties of the underlying structural model, rather than being assumed to hold a priori.  \\ \vspace*{.1cm}

			Causal graphical models thus combine  the  accessibility  and  flexibility  of  potential  outcomes  with  the  preciseness and  analytical  rigor  of  structural  econometrics" (\cite{Hunermund2021})
		\end{frame}

		\begin{frame}{structural causal models (SCM)}
			Parameters of a probabilistic structural model can be estimated with a \href{https://github.com/stan-dev/rstan/wiki/RStan-Getting-Started}{\underline{\smash{Bayesian}}} \href{https://www.bnlearn.com/}{\underline{\smash{network}}} \href{https://github.com/paul-buerkner/brms}{\underline{\smash{approach}}} ...
			\begin{center}
				\includegraphics[scale=.175]{Bayesian-network.png}
				\captionof{figure}{Bayesian network. Source: \cite{Melendez2019}}
			\end{center}
			... \href{https://cran.r-project.org/web/packages/cSEM/vignettes/cSEM.html}{\underline{\smash{partial least squares}}}, or through \href{https://github.com/VC2015/DMLonGitHub}{\underline{\smash{double debiased machine learning}}} (cf. cite{Chernozhukov2018,Jung2021})
		\end{frame}

		\begin{frame}{The do-calculus}
			\begin{center}
				\includegraphics[scale=.3275]{CausalHierarchy2.jpg}
				\captionof{figure}{The causal hierarchy. Source: \cite{Bareinboim2020}}
			\end{center}
		\end{frame}

% ------------------------------------------------------------------------------
\section{Conclusion}


	\begin{frame}{Wrapping it all up}
		\begin{center}
			\includegraphics[scale=3]{WrapItUp.jpg}
		\end{center}
	\end{frame}

	\begin{frame}{Wrapping it all up}
		To estimate a causal effect, we can compare the outcome to its counterfactual.
		\begin{itemize}
			\item<1-> potential outcomes: $ \tau  =  E \{\textcolor{red}{Y}_i(\textcolor{blue}{1})\} - E \{\textcolor{red}{Y}_i(\textcolor{blue}{0})\} $
			\begin{itemize}
				\item<2-> RCTs
				\item<2-> DID \& 2WFE
				\item<2-> Matching \& synthetic controls
				\item<2-> IV \& RDD
				\item<2-> causal trees
				\item<2->	SCM
			\end{itemize}
			\item<3-> with frequentist, Bayesian, or machine learning methods
			\item<4-> it should be conceptually sound and methodologically robust
			\item<5> I think the relation with theory should be healthy (inductively so or deductive)
		\end{itemize}
	\end{frame}

% ------------------------------------------------------------------------------
\section{further reads}

	\begin{frame}{sources}
		Beyond the coursebooks, see
		\begin{itemize}
			\item James, Witten, Hastie and Tibshirani (\citeyear{James2013})
			\begin{itemize}
				\item introduction to statistical learning (\href{https://www.edx.org/course/statistical-learning}{\underline{\smash{edX}}})
			\end{itemize}
			\item Boehmke and Greenwell (2020)
			\begin{itemize}
				\item \href{https://bradleyboehmke.github.io/HOML/}{\underline{\smash{Hands-On Machine Learning with R}}}
			\end{itemize}
			\item Paul Goldsmith-Pinkham 2021
			\begin{itemize}
				\item \href{https://github.com/paulgp/applied-methods-phd}{\underline{\smash{applied methods PhD course}}}
			\end{itemize}
			\item Nick Huntington-Klein 2021
			\begin{itemize}
				\item \href{https://nickchk.com/causalitybook.html}{\underline{\smash{The Effect: An Introduction to Research Design and Causality}}}
			\end{itemize}
			\item \cite{Peters2017}
			\begin{itemize}
				\item Elements of causal inference: foundations and learning algorithms
			\end{itemize}
			\item \cite{Nagarajan2013}
			\begin{itemize}
				\item Bayesian Networks in R
			\end{itemize}
		\end{itemize}
	\end{frame}

	\begin{frame}{deciding on literature}
		\begin{center}
			\includegraphics[width = .6\textwidth]{WhatCausalInferenceBookShouldIRead.jpg}
			\captionof{figure}{decision tree, source: \href{https://www.bradyneal.com/which-causal-inference-book}{\underline{\smash{Brady Neal 2020}}}}
		\end{center}
	\end{frame}



% ------------------------------------------------------------------------------
	\begin{frame}[t, allowframebreaks]{References}
	% [t,allowframebreaks]
	  \printbibliography
	\end{frame}
% ------------------------------------------------------------------------------

%
\end{document}
